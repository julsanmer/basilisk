\section{User Guide}
The three possible input messages are:
\begin{itemize}
	\item {\tt scStateInMsg}: spacecraft state input message
	\item {\tt spiceStateInMsg}: spice state input message
	\item {\tt elemInMsg}: classical orbit elements input message
\end{itemize}
Note that only one can be connected to.  

The three output messages are:
\begin{itemize}
	\item {\tt scStateInMsg}: spacecraft state input message
	\item {\tt spiceStateInMsg}: spice state input message
	\item {\tt elemInMsg}: classical orbit elements input message
\end{itemize}
Here you can connect to any of these messages.  This makes it possible to take a single input and convert into multiple output formats at the same time.

The gravitational constant $\mu$ must be defined in all cases.
\begin{verbatim}
	orb_elemObject = orbElemConvert.OrbElemConvert()
	orb_elemObject.ModelTag = "OrbElemConvertData"
	orb_elemObject.mu = mu
\end{verbatim}


\begin{thebibliography}{1}
	\bibitem{bib:1}
	Vallado, D. A., and McClain, W. D., \textit{Fundamentals of Astrodynamics and Applications, 4th ed}. Hawthorne, CA: Published by Microcosm Press, 2013.
	\bibitem{bib:2}
	Schaub, H., and Junkins, J. L., \textit{Analytical Mechanics of Space Systems, 3rd ed.}. Reston, VA: American Institute of Aeronautics and Astronautics.
\end{thebibliography}