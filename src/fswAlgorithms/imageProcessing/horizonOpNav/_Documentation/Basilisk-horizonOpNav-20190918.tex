% README file for moduleDocumentationTemplate TeX template.
% This template should be used to document all Basilisk modules.
% Updated 20170711 - S. Carnahan
%
%-Copy the contents of this folder to your own _Documentation folder
%
%-Rename the Basilisk-moduleDocumentationTemplate.tex appropriately
%
% All edits should be made in one of:
% sec_modelAssumptionsLimitations.tex
% sec_modelDescription.tex
% sec_modelFunctions.tex
% sec_revisionTable.tex
% sec_testDescription.tex
% sec_testParameters.tex
% sec_testResults.tex
% sec_user_guide.tex
%
% NOTE: if the TeX document is reading in auto-generated TeX snippets from the AutoTeX folder, then
%            pytest must first be run for the unit test of this module.  This process creates the required unit test results 
%.           that are read into this document.  
%
%-Some rules about referencing within the document:
%1. If writing the user guide, assume the module description is present
%2. If writing the validation section, assume the module features section is present
%3. Make no other assumptions about any sections being present. This allow for sections of the document to be used elsewhere without breaking.

%In order to import some of these sections into a document in a different directory:
%\usepackage{import}
%Then, the sections are called with \subimport{relative path}{file} in order to \input{file} using the right relative path.
%\import{full path}{file} can also be used if absolute paths are preferred over relative paths.

%%%%%%%%%%%%%%%%%%%%%%%%%%%%%%%%%%%%%%%%%%%%%%%%%




\documentclass[]{BasiliskReportMemo}

\usepackage{cite}
\usepackage{AVS}
\usepackage{float} %use [H] to keep tables where you put them
\usepackage{array} %easy control of text in tables
\usepackage{graphicx}
\bibliographystyle{plain}


\newcommand{\submiterInstitute}{Autonomous Vehicle Simulation (AVS) Laboratory,\\ University of Colorado}


\newcommand{\ModuleName}{horizonOpNav}
\newcommand{\subject}{Limb-Based Optical Navigation Module}
\newcommand{\status}{Draft}
\newcommand{\preparer}{T. Teil}
\newcommand{\summary}{Converter that takes circle information from image processing and turns it into a inertial position. This module also maps the uncertainty from center and apparent diameter into a position uncertainty.}

\begin{document}

\makeCover

%
%	enter the revision documentation here
%	to add more lines, copy the table entry and the \hline, and paste after the current entry.
%
\pagestyle{empty}
{\renewcommand{\arraystretch}{2}
\noindent
\begin{longtable}{|p{0.5in}|p{3.5in}|p{1.07in}|p{0.9in}|}
\hline
{\bfseries Rev} & {\bfseries Change Description} & {\bfseries By}& {\bfseries Date} \\
\hline
1.0 & Initial Release & T. Teil & 2019-05-27\\
\hline

\end{longtable}
}



\newpage
\setcounter{page}{1}
\pagestyle{fancy}

\tableofcontents %Autogenerate the table of contents
~\\ \hrule ~\\ %Makes the line under table of contents









	
% !TEX root = ./Basilisk-inertialUKF-20190402.tex

\section{Model Description}

This module implements a square-root unscented Kalman Filter in order to achieve it's best state estimate of the inertial spacecraft attitude states. The estimated state is the attitude (MRPs) and the spacecraft rotation rate in the body frame. 

\subsection{Filter Setup} %%%

The equations and algorithm for the square root uKF are given in "inertialUKF$\_$DesignBasis.pdf" [\citenum{Wan2001}] alongside this document.

The filter is therefore derived with the states being $\bm X =\begin{bmatrix} \bm \sigma_{\mathcal{B}/\mathcal{N}}& \bm \omega_{\mathcal{B}/\mathcal{N}} \end{bmatrix}^{T}$

The dynamics of the filter are given in Equations \eqref{eq:dynInertial}. $\tau$ is the total torque read in by the wheels. 
\begin{align}
\label{eq:dynInertial}
\dot{\bm \sigma} &= \frac{1}{4} [B] \bm \omega_{\mathcal{B}/\mathcal{N}} \\
\dot{\bm \omega}_{\mathcal{B}/\mathcal{N}} & = [I]^{-1} \tau
\end{align}

The following square-root uKF coefficients are used: $\alpha = 0.02$, and $\beta = 2$. 


\subsection{Measurements}

The measurement model is given in equation \ref{eq:meas}. Since the input MRP may or may not be in the same "shadow" set as the state estimate, they are assured to be in the same representation. This prevents from getting residuals of 360\dg. 

This is done following these steps:

\begin{itemize}
\item Current state estimate and measurements turned to quaternions
\item State estimate is transposed (scaled by -1)
\item Both quaternions are added and the sum turned to an MRP
\item If the sum is greater than one the MRPs were not in the same representation and the measurement is shadowed
\end{itemize}

\begin{equation}\label{eq:meas}
\bm G_i(\bm X) = \bm \sigma
\end{equation}

%\subsection{Measurements}
%
%The measurement model is given in equation \ref{eq:meas}. Since the input MRP may or may not be in the same "shadow" set as the state estimate, they are assured to be in the same representation. This prevents from getting residuals of 360\dg. 
%
%This is done following these steps:
%
%\begin{itemize}
%\item Current state estimate and measurements turned to quaternions
%\item State estimate is transposed (scaled by -1)
%\item Both quaternions are added and the sum turned to an MRP
%\item If the sum is greater than one the MRPs were not in the same representation and the measurement is shadowed
%\end{itemize}
%
%\begin{equation}\label{eq:meas}
%\bm G_i(\bm X) = \bm \sigma
%\end{equation} %This section includes mathematical models, code description, etc.

% !TEX root = ./Basilisk-rwNullspace-20190209.tex


\section{Module Functions}
This module has the following functions:
\begin{itemize}
	\item \textbf{Evaluate RW null projection matrix $[\tau]$}: When reset the module will pull in the current RW configuration data and create the null motion projection matrix.  This matrix remains fixed unit the module is reset again.
	\item \textbf{Compute a RW deceleration torque}: With each update call the module computes a decelerating RW torque solution that lies in the null space of the RW array.
	\item \textbf{Output a net RW motor torque solution}:  The module combined the feedback control torque and the null space torque to slow down the RW speeds and outputs a net solution solution.
\end{itemize}

\section{Module Assumptions and Limitations}
The module assumes all RW devices are operating and available.  It also assumes the RW spin axes don't change during the regular update cycles.   %This includes a concise list of what the module does. It also includes model assumptions and limitations

% !TEX root = ./Basilisk-MRPROTATION-20180522.tex

\section{Test Description and Success Criteria}
The module is run on its own with specified inputs to ensure the outputs are correct.  The outputs are evaluated dynamically using a support python script, and then compared to the Basilisk evaluated results.  A nominal simulation length of 1 second is used with a time step of 0.5 seconds, yielding 3 return values.

If the rotation states are set directly in the module by specifying {\tt mrpSet} and {\tt omega\_RR0\_R}, then the values $\bm\sigma_{R/R_{0}}(t_{0}) = $[0.3 0.5 0. ] and $\leftexp{R}{\bm\omega}_{R/R_{0}}=$[0.1 0.  0. ]deg/sec are used.  The simulation flag {\tt cmdStateFlag} determines if the rotation states are specified through an input message.  If yes, then the values $\bm\sigma_{R/R_{0}}(t_{0}) = $[ 0.1  0.  -0.2] and $\leftexp{R}{\bm\omega}_{R/R_{0}}=$[0.1 1.  0.5]deg/sec are used instead.  

If the simulation flag {\tt stateOutputFlag} is true then  the optional attitude states of $\mathcal{R}$ relative to $\mathcal{R}_{0}$ are provided in an output message.

If the simulation flag {\tt testReset} is true then  the simulation will run an addition 1 second, but after a reset function is called.

\begin{table}[htbp]
	\caption{Test Scenarios.}
	\label{tab:checks}
	\centering \fontsize{10}{10}\selectfont
	\begin{tabular}{ l | c | c | c } % Column formatting, 
		\hline\hline
		\textbf{Check}  & {\tt cmdStateFlag} & {\tt testReset}  \\ 
		\hline
		1 & False  & False \\
		2 & True  & False \\
		3 & False  & True \\
		4 & True  & True \\
		\hline\hline
	\end{tabular}
\end{table}

\section{Test Parameters}
The output variables being tested are listed in Table~\ref{tab:errortol}, including the test tolerance value.

\begin{table}[htbp]
	\caption{Error tolerance for each test.}
	\label{tab:errortol}
	\centering \fontsize{10}{10}\selectfont
	\begin{tabular}{ c | c } % Column formatting, 
		\hline\hline
		\textbf{Output Value Tested}  & \textbf{Tolerated Error}  \\ 
		\hline
		{\tt attRefOutMsg.sigma\_RN}        & 1e-10	   \\
		{\tt attRefOutMsg.omega\_RN\_N}        & 1e-10	   \\
		{\tt attRefOutMsg.domega\_RN\_N}        & 1e-10	   \\
		{\tt attitudeOutMsg.state}        & 1e-10	   \\
		{\tt attitudeOutMsg.rate}        & 1e-10	   \\
		\hline\hline
	\end{tabular}
\end{table}




\section{Test Results}
The results of the unit test are listed in Table~\ref{tab:results}.  
All of the tests are expected to pass:
\begin{table}[H]
	\caption{Test results}
	\label{tab:results}
	\centering \fontsize{10}{10}\selectfont
	\begin{tabular}{c | c  } % Column formatting, 
		\hline\hline
		\textbf{Check} 						  		&\textbf{Pass/Fail} \\ 
		\hline
	   1	   			& \textcolor{ForestGreen}{PASSED} \\ 
	   2	   			& \textcolor{ForestGreen}{PASSED} \\ 
	   3	   			& \textcolor{ForestGreen}{PASSED} \\ 
	   4	   			& \textcolor{ForestGreen}{PASSED} \\ 
	   \hline\hline
	\end{tabular}
\end{table}


 % This includes test description, test parameters, and test results

% !TEX root = ./Basilisk-inertialUKF-20190402.tex

\section{User Guide}
\subsection{Filter Set-up, initialization, and I/O}

In order for the filter to run, the user must set a few parameters:

\begin{itemize}
\item The unscented filter has 3 parameters that need to be set, and are best as: \\
      \texttt{ filterObject.alpha = 0.02} \\
      \texttt{ filterObject.beta = 2.0} \\
      \texttt{ filterObject.kappa = 0.0} 
\item Initialize orbit: \\ 
\texttt{     mu = 42828.314*1E9 \#m3/s2} \\
 \texttt{    elementsInit = orbitalMotion.ClassicElements()} \\
 \texttt{    elementsInit.a = 4000*1E3 \#meters} \\
 \texttt{    elementsInit.e = 0.2} \\
 \texttt{    elementsInit.i = 10} \\
  \texttt{   elementsInit.Omega = 0.001} \\
 \texttt{    elementsInit.omega = 0.01} \\
 \texttt{    elementsInit.f = 0.1} \\
 \texttt{    r, v = orbitalMotion.elem2rv(mu, elementsInit)} 
\item The initial covariance: \\
 \texttt{Filter.covar =} \\
  \texttt{[1000*1E6, 0.0, 0.0, 0.0, 0.0, 0.0,\\
                              0.0, 1000.*1E6, 0.0, 0.0, 0.0, 0.0,\\
                              0.0, 0.0, 1000.*1E6, 0.0, 0.0, 0.0,\\
                              0.0, 0.0, 0.0, 5.*1E6, 0.0, 0.0,\\
                              0.0, 0.0, 0.0, 0.0, 5.*1E6, 0.0,\\
                              0.0, 0.0, 0.0, 0.0, 0.0, 5.*1E6]}
 \item The initial state :\\
  \texttt{      filterObject.stateInit = r.tolist() + v.tolist()} 
    \item The process noise :\\
  \texttt{     qNoiseIn = np.identity(6)} \\
  \texttt{     qNoiseIn[0:3, 0:3] = qNoiseIn[0:3, 0:3]*1E-8*1E-8} \\
  \texttt{     qNoiseIn[3:6, 3:6] = qNoiseIn[3:6, 3:6]*1E7*1E7} \\
  \texttt{     filterObject.qNoise = qNoiseIn.reshape(36).tolist()}
\end{itemize}

 % Contains a discussion of how to setup and configure  the BSK module






\bibliography{bibliography} %This includes references used and mentioned.

\end{document}
