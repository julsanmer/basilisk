% README file for making a technical note for a module
%



\documentclass[]{BasiliskReportMemo}

\usepackage{cite}
\usepackage{AVS}
\usepackage{float} %use [H] to keep tables where you put them
\usepackage{array} %easy control of text in tables
\usepackage{graphicx}
\bibliographystyle{plain}


\newcommand{\submiterInstitute}{Autonomous Vehicle Simulation (AVS) Laboratory,\\ University of Colorado}


\newcommand{\ModuleName}{Module Name}
\newcommand{\status}{Status}

\begin{document}

\makeCover

%
%	enter the revision documentation here
%	to add more lines, copy the table entry and the \hline, and paste after the current entry.
%
\pagestyle{empty}
{\renewcommand{\arraystretch}{2}
\noindent
\begin{longtable}{|p{0.5in}|p{3.5in}|p{1.07in}|p{0.9in}|}
\hline
{\bfseries Rev} & {\bfseries Change Description} & {\bfseries By}& {\bfseries Date} \\
\hline
1.0 & Revision Description & F. Last1 & YYYYMMDD\\
\hline
1.1 & Revision Description& F. Last2 & YYYYMMDD\\
\hline

\end{longtable}
}



\newpage
\setcounter{page}{1}
\pagestyle{fancy}

\tableofcontents %Autogenerate the table of contents
~\\ \hrule ~\\ %Makes the line under table of contents



\section{Introduction}
It is preferred to include the associated math with the module {\tt doxygen} documentation in the module {\tt .h} file.  However, if the associated math is too extensive it is possible to include a brief technical note using this TeX template.



\section{Alternate Citation Options}
Instead of writing a technical note, if the module math is presented in a conference paper, journal paper, or MS thesis or PhD dissertation, it is simple to link to such a document from within the module {\tt .h} file.  The {\tt doxygen} command to cite an external web link is
\begin{verbatim}
	[link name](http://example.link)
\end{verbatim}
 We are trying to keep the Basilisk repository as small as possible, thus using a PDF technical note should be used sparingly.  








\bibliography{bibliography} %This includes references used and mentioned.

\end{document}
