% !TEX root = ./Basilisk-oeStateEphem-20190426.tex

\section{Test Description and Success Criteria}

\subsection{Check 1: time measure outside of Chebyshev time interval}
The unit test makes sure that the correct response is obtained if a time step is used that is outside of the Chebyshev time intervals.  This tests the expected failing high and low feature.  The test is expected to pass.


\subsection{Check 2: TDRSS Data Fit }
The Spice trajectory of the TDRSS satellite is loaded.  The trajectory is mapped into orbit elements which are then converted to Chebyshev coefficients.  The simulation is run for 4 days and the compute trajectory is then compared to the actual Spice TDRSS trajectory.  Both the inertial position and velocity vectors are checked.




\section{Test Parameters}

The Spice TDRSS trajectory data is loaded as the truth trajectory.  A 14th order Chebyshev model is then created and uploaded to the BSK module.  
The unit tests verify that the module output guidance message vectors match expected values.
\begin{table}[htbp]
	\caption{Error tolerance for each test.}
	\label{tab:errortol}
	\centering \fontsize{10}{10}\selectfont
	\begin{tabular}{ c | c } % Column formatting, 
		\hline\hline
		\textbf{Output Value Tested}  & \textbf{Tolerated Error}  \\ 
		\hline
		{\tt r\_BdyZero\_N}        & 10000.0 m	   \\ 
		{\tt v\_BdyZero\_N}        & 1.0 m/s   \\ 
		\hline\hline
	\end{tabular}
\end{table}




\section{Test Results}
All of the tests passed:
\begin{table}[H]
	\caption{Test results}
	\label{tab:results}
	\centering \fontsize{10}{10}\selectfont
	\begin{tabular}{c | c  } % Column formatting, 
		\hline\hline
		\textbf{Check} 						  		&\textbf{Pass/Fail} \\ 
		\hline
	   1	   			& \textcolor{ForestGreen}{PASSED} \\ 
	   2	   			& \textcolor{ForestGreen}{PASSED} \\ 
	   \hline\hline
	\end{tabular}
\end{table}





