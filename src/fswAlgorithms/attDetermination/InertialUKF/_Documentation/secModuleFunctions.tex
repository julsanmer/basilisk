% !TEX root = ./Basilisk-inertialUKF-20190402.tex


\section{Module Functions}

\begin{itemize}
    \item \textbf{Read ST Messages: } Read in the messages from all available star trackers and orders them with respect to time of measurement. 
    \item \textbf{Inertial UKF Agg Gyr Data: } Aggregate the input gyro data into a combined total quaternion 
    rotation to push the state forward.  This information is stored in the 
    main data structure for use in the propagation routines.
    \item \textbf{Inertial UKF Time Update: } Performs the filter time update as defined in the baseline algorithm
    \item \textbf{Inertial UKF Meas Update: } Performs the filter measurement update as defined in the baseline algorithm
    \item \textbf{Inertial UKF Meas Model: } Predicts the measurements given current state and measurement model $\bm G$
    \item \textbf{inertial State Prop: } Integrates the state given the $\bm F$ dynamics of the system
    \item \textbf{Inertial Clean Update: } Returns filter to a previous state in the case of a bad computation
    \end{itemize}

\section{Module Assumptions and Limitations}

The assumptions of this module are all tied in to the underling assumptions and limitations to a working filter. 
In order for a proper convergence of the filter, the dynamics need to be representative of the actual spacecraft perturbations.
Depending on the tuning of the filter (process noise value and measurement noise value), the robustness of the solution will be weighed against it's precision. 

The number of measurements and the frequency of their availability also influences the general performance. 

