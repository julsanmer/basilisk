\section{Unit Test Description}
A series of unit tests are performed to check the validity of this module's operation.

\subsection{Test 1}
The first test uses an input vector of $\bm u_s = [0.05, 0.0, -0.15, -0.2]$ Nm.  The RW spin inertia is set to $J_{s}$ = 0.1 kg m$^{s}$, while the maximum RW motor torque is set to $u_{\text{max}}$ = 0.2 Nm.  In this test case no RW availability or wheel speed messages are set.  The simulation is first run for 1.5 seconds with a 0.5 second control update period.  Next, the module is reset and run for another 1.5 sections.  With only the open-loop voltage conversion active and the RW motor torque
The resulting actual values and differences with hand-computed values are shown in Table~\ref{tbl:testFalseFalseFalse}.  The reset of the module should have not impact on the voltage conversion, which is the case.


\subsection{Test 2}
This test repeats the values of test 1, except that the RW motor torque input vector is set to $\bm u_s = [0.5, 0.0, -0.15, -0.5]$ Nm.  This should saturate the first and last RW voltage output, which is seen in Table~\ref{tbl:testTrueFalseFalse}. 


\subsection{Test 3}
This test repeats the values of test 1, except that a RW availability message is created.  Here all RWs have a status of {\tt AVAILABLE} except for the 3rd RW which is {\tt UNAVAILALBE}.    The 3rd RW voltages should thus all be 0.0 in this case, which is seen in Table~\ref{tbl:testFalseTrueFalse}. 


\subsection{Test 4}

This test repeats the values of test 1, except that a RW wheel speed message is created.  The feedback gain is set to $K = 1.5$.  For the first 1.0 seconds the RW wheel speeds are set to $\bm\Omega = [1.0, 2.0, 1.5, -3.0]$ rad/sec.  At 1.0 seconds the RW wheel speeds are set to  $\bm\Omega = [1.1, 2.1, 1.1, -4.1]$ rad/sec.  Then the module is reset at 1.5 seconds and the simulation continued for another 1.5 second for a 3 second total simulation time.   The speed message remains the same after the reset.   Table~\ref{tbl:testFalseFalseTrue} shows the results of the actual values, and the differences with the hand-computed values.  


\section{Test Parameters}
Test parameters are interspersed with the test description above. For all tests, $V_{\mathrm{min}} = 1$ and $V_{\mathrm{max}} = 11$ which correspond to a maximum torque of 0.2 [Nm].

\section{Test Results}
Results for each test are shown in the tables below:
\begin{table}[htbp]\caption{RW voltage output for case {\tt useLargeVoltage = False, useAvailability = False, useTorqueLoop = False}.}\label{tbl:testFalseFalseFalse}\centering\begin{tabular}{ccccccccc}
\hline
  time [s]  &  $V_{s,1}$  &  Error  &  $V_{s,2}$  &  Error  &  $V_{s,3}$  &  Error  &  $V_{s,4}$  &  Error  \\
\hline
     0      &     3.5     &    0    &      0      &    0    &    -8.5     &    0    &     -11     &    0    \\
    0.5     &     3.5     &    0    &      0      &    0    &    -8.5     &    0    &     -11     &    0    \\
     1      &     3.5     &    0    &      0      &    0    &    -8.5     &    0    &     -11     &    0    \\
    1.5     &     3.5     &    0    &      0      &    0    &    -8.5     &    0    &     -11     &    0    \\
     2      &     3.5     &    0    &      0      &    0    &    -8.5     &    0    &     -11     &    0    \\
    2.5     &     3.5     &    0    &      0      &    0    &    -8.5     &    0    &     -11     &    0    \\
     3      &     3.5     &    0    &      0      &    0    &    -8.5     &    0    &     -11     &    0    \\
\hline\end{tabular}\end{table}
\begin{table}[htbp]\caption{RW voltage output for case {\tt useLargeVoltage = True, useAvailability = False, useTorqueLoop = False}.}\label{tbl:testTrueFalseFalse}\centering\begin{tabular}{ccccccccc}
\hline
  time [s]  &  $V_{s,1}$  &  Error  &  $V_{s,2}$  &  Error  &  $V_{s,3}$  &  Error  &  $V_{s,4}$  &  Error  \\
\hline
     0      &     11      &    0    &      0      &    0    &    -8.5     &    0    &     -11     &    0    \\
    0.5     &     11      &    0    &      0      &    0    &    -8.5     &    0    &     -11     &    0    \\
     1      &     11      &    0    &      0      &    0    &    -8.5     &    0    &     -11     &    0    \\
    1.5     &     11      &    0    &      0      &    0    &    -8.5     &    0    &     -11     &    0    \\
     2      &     11      &    0    &      0      &    0    &    -8.5     &    0    &     -11     &    0    \\
    2.5     &     11      &    0    &      0      &    0    &    -8.5     &    0    &     -11     &    0    \\
     3      &     11      &    0    &      0      &    0    &    -8.5     &    0    &     -11     &    0    \\
\hline\end{tabular}\end{table}
\begin{table}[htbp]\caption{RW voltage output for case {\tt useLargeVoltage = False, useAvailability = True, useTorqueLoop = False}.}\label{tbl:testFalseTrueFalse}\centering\begin{tabular}{ccccccccc}
\hline
  time [s]  &  $V_{s,1}$  &  Error  &  $V_{s,2}$  &  Error  &  $V_{s,3}$  &  Error  &  $V_{s,4}$  &  Error  \\
\hline
     0      &     3.5     &    0    &      0      &    0    &      0      &    0    &     -11     &    0    \\
    0.5     &     3.5     &    0    &      0      &    0    &      0      &    0    &     -11     &    0    \\
     1      &     3.5     &    0    &      0      &    0    &      0      &    0    &     -11     &    0    \\
    1.5     &     3.5     &    0    &      0      &    0    &      0      &    0    &     -11     &    0    \\
     2      &     3.5     &    0    &      0      &    0    &      0      &    0    &     -11     &    0    \\
    2.5     &     3.5     &    0    &      0      &    0    &      0      &    0    &     -11     &    0    \\
     3      &     3.5     &    0    &      0      &    0    &      0      &    0    &     -11     &    0    \\
\hline\end{tabular}\end{table}
\begin{table}[htbp]\caption{RW voltage output for case {\tt useLargeVoltage = False, useAvailability = False, useTorqueLoop = True}.}\label{tbl:testFalseFalseTrue}\centering\begin{tabular}{ccccccccc}
\hline
  time [s]  &  $V_{s,1}$  &    Error     &  $V_{s,2}$  &    Error     &  $V_{s,3}$  &  Error  &  $V_{s,4}$  &    Error     \\
\hline
     0      &     3.5     &      0       &      0      &      0       &    -8.5     &    0    &     -11     &      0       \\
    0.5     &     3.5     &      0       &      0      &      0       &    -8.5     &    0    &     -11     &      0       \\
     1      &     3.5     &      0       &      0      &      0       &    -8.5     &    0    &     -11     &      0       \\
    1.5     &    5.75     & -1.77636e-15 &    -2.5     & -1.33227e-15 &     -11     &    0    &    -9.5     & -5.32907e-15 \\
     2      &     3.5     &      0       &      0      &      0       &    -8.5     &    0    &     -11     &      0       \\
    2.5     &     3.5     &      0       &      0      &      0       &    -8.5     &    0    &     -11     &      0       \\
     3      &    7.25     &      0       &      0      &      0       &     -11     &    0    &     -11     &      0       \\
\hline\end{tabular}\end{table}

All of the tests passed:
\begin{table}[H]
	\caption{Test results}
	\label{tab:results}
	\centering \fontsize{10}{10}\selectfont
	\begin{tabular}{c | c  } % Column formatting, 
		\hline
		\textbf{Test} 						  		&\textbf{Pass/Fail} \\ \hline
	   1	   			& \textcolor{ForestGreen}{PASSED} \\ \hline
		2	   			& \textcolor{ForestGreen}{PASSED} \\ \hline
		3	   			& \textcolor{ForestGreen}{PASSED} \\ \hline
		4	   			& \textcolor{ForestGreen}{PASSED} \\ \hline
	\end{tabular}
\end{table}





\pagebreak %needed to keep images/paragraphs in the right place. Cannot \usepackage{float} here because it is not used in the AutoTex implementation.