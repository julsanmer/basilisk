% !TEX root = ./Basilisk-dvAccumulation-2019-03-28.tex

\section{Test Description and Success Criteria}
The unit test creates an input message with time tagged accelerometer measurements.  




\section{Test Parameters}

Test and simulation parameters and inputs go here. Basically, describe your test in the section above, but put any specific numbers or inputs to the tests in this section.  The test simulation period is 2 seconds with a 0.5 second time step.  

The unit test verifies that the module output navigation message vectors match expected values.
\begin{table}[htbp]
	\caption{Error tolerance for each test.}
	\label{tab:errortol}
	\centering \fontsize{10}{10}\selectfont
	\begin{tabular}{ c | c } % Column formatting, 
		\hline\hline
		\textbf{Output Value Tested}  & \textbf{Tolerated Error}  \\ 
		\hline
		{\tt vehAccumDV}        & 1e-10	   \\ 
		{\tt timeTag}        & 1e-10	   \\ 
		\hline\hline
	\end{tabular}
\end{table}




\section{Test Results}
All of the tests passed:
\begin{table}[H]
	\caption{Test results}
	\label{tab:results}
	\centering \fontsize{10}{10}\selectfont
	\begin{tabular}{c | c  } % Column formatting, 
		\hline\hline
		\textbf{Check} &\textbf{Pass/Fail} \\ 
		\hline
	   1	   			& \textcolor{ForestGreen}{PASSED} \\ 
	   \hline\hline
	\end{tabular}
\end{table}


