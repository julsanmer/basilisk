% !TEX root = ./Basilisk-inertialUKF-20190402.tex

\section{User Guide}
\subsection{Filter Set-up, initialization, and I/O}

In order for the filter to run, the user must set a few parameters:

\begin{itemize}
\item The unscented filter has 3 parameters that need to be set, and are best as: \\
      \texttt{ filterObject.alpha = 0.02} \\
      \texttt{ filterObject.beta = 2.0} \\
      \texttt{ filterObject.kappa = 0.0} 
\item The star trackers: \\ 
  \texttt{    ST1Data = inertialUKF.STMessage()}\\
    \texttt{  ST1Data.stInMsgName = "star$\_$stracker$\_$1$\_$data"}\\
  \texttt{    ST1Data.noise = [0.00017 * 0.00017, 0.0, 0.0,\\
                         0.0, 0.00017 * 0.00017, 0.0,\\
                         0.0, 0.0, 0.00017 * 0.00017]}\\
   \texttt{   ST2Data = inertialUKF.STMessage()}\\
  \texttt{    ST2Data.stInMsgName = "star$\_$tracker$\_$2$\_$data"}\\
    \texttt{  ST2Data.noise = [0.00017 * 0.00017, 0.0, 0.0,\\
                         0.0, 0.00017 * 0.00017, 0.0,\\
                         0.0, 0.0, 0.00017 * 0.00017]}\\
  \texttt{    STList = [ST1Data, ST2Data]}\\
   \texttt{   filterObject.STDatasStruct.STMessages = STList}\\
   \texttt{   filterObject.STDatasStruct.numST = len(STList)}\\
\item The initial covariance: \\
 \texttt{Filter.covar =} \\
  \texttt{ [1., 0.0, 0.0, 0.0, 0.0, \\
                          0.0, 1., 0.0, 0.0, 0.0,\\
                          0.0, 0.0, 1., 0.0, 0.0,\\
                          0.0, 0.0, 0.0, 0.02, 0.0,\\
                          0.0, 0.0, 0.0, 0.0, 0.02]}
 \item The initial state :\\
 \texttt{Filter.state =[0.0, 0.0, 1.0, 0.0, 0.0]}
  \item The low pass filter for the accelerometers :\\
    \texttt{ lpDataUse = inertialUKF.LowPassFilterData()}\\
   \texttt{ lpDataUse.hStep = 0.5 }\\
 \texttt{   lpDataUse.omegCutoff = 15.0/(2.0*math.pi) }\\
 \texttt{   filterObject.gyroFilt = [lpDataUse, lpDataUse, lpDataUse] }\\
\end{itemize}



