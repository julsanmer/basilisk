% !TEX root = ./Basilisk-magFieldDipole-20190309.tex

\section{User Guide}

\subsection{General Module Setup}
This section outlines the steps needed to add a {\tt MagneticField} module to a sim.
First, the planet magnetic field model must be imported and initialized:
\begin{verbatim}
from Basilisk.simulation import magneticFieldCenteredDipole
magModule = magneticFieldCenteredDipole.MagneticFieldCenteredDipole()
magModule = "CenteredDipole"
\end{verbatim}
By default the model the dipole parameters are zeroed, resulting in a zeroed magnetic field message. 

The model can  be added to a task like other simModels. 
\begin{verbatim}
unitTestSim.AddModelToTask(unitTaskName, testModule)
\end{verbatim}

Each {\tt MagneticField} module calculates the magnetic field based on the output state messages for a set of spacecraft.
To add spacecraft to the model the spacecraft state output message name is sent to the \verb|addScToModel| method:
\begin{verbatim}
scObject = spacecraft.Spacecraft()
scObject.ModelTag = "spacecraftBody"
magModule.addSpacecraftToModel(scObject.scStateOutMsg)
\end{verbatim}

\subsection{Planet Ephemeris Information}
The optional planet state message name can be set by directly adjusting that attribute of the class:
\begin{verbatim}
magModule.planetPosInMsg.subscribeTo(planetMsg)
\end{verbatim}
If SPICE is not being used, the planet is assumed to reside at the origin and $\bm r_{P/O} = \bm 0$.

\subsection{Setting the Model Reach}
By default the model doesn't perform any checks on the altitude to see if the specified magnetic field model should be used.  This is set through the parameters {\tt envMinReach} and {\tt envMaxReach}.  Their default values are -1.  If these are set to positive values, then if the spacecraft orbit radius is smaller than {\tt envMinReach} or larger than {\tt envMaxReach}, the magnetic field vector is set to zero.


\subsection{Centered Dipole Magnetic Parameters}
The parameters of the dipole model are set by calling
\begin{verbatim}
    magModule.g10 = g10Value
    magModule.g11 = g11Value
    magModule.h11 = h11Value
    magModule.planetRadius = planetRadius # in meters
\end{verbatim}
where $g_{1}^{0}$, $g_{1}^{1}$ and $h_{1}^{1}$ are the first three expansion terms of the IGRF spherical harmonics model.\footnote{\url{https://www.ngdc.noaa.gov/IAGA/vmod/igrf.html}}

The python support file {\tt simSetPlanetEnvironment.py} contains a helper function called 
\begin{verbatim}
	centeredDipoleMagField()
\end{verbatim} 
which helps setup common NASA centered dipole models for a range of planets that contain global magnetic fields.  This possible planet names includes mercury, earth, jupiter, saturn, uranus and neptune.  The function is then called with
\begin{verbatim}
simSetPlanetEnvironment.centeredDipoleMagField(testModule, "jupiter")
\end{verbatim}
to setup the named planets dipole model.

