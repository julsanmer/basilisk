% !TEX root = ./Basilisk-rwNullspace-20190209.tex




\section{Module Description}
\subsection{Module Purpose}
This module aims to drive the reaction wheel (RW) speeds to desired values by using the null space of the RW array.  The input and output messages of {\tt rwNullSpace} are shown in Figure~\ref{fig:moduleImg}.  There are three required input messages:
\begin{enumerate}
	\item The feedback control torque which is mapped onto the RW motor torque solution space.  The message type is {\tt RWArrayTorqueIntMsg}, the same message type as the module output message.
	\item The RW speed array message of type {\tt RWSpeedIntMsg}.
	\item The RW configuration message contains the RW spin axes information. The message type is {\tt RWConstellationFswMsg}.
\end{enumerate}
There is one optional input message which provides the desired RW speeds.  If this message is not connected, then the desired speeds default to zero values.

The module output message is a RW motor torque array message that sums the input control torque solution with the new null space torque solution that will reduce the RW speeds.  


\subsection{RW Null Space Mathematics}
Let $N$ be the number of RWs present, while $\hat{\bm g}_{s_{i}}$ is the $i^{\text{th}}$ RW spin axis unit direction vector.  The RW spin axes matrix is defined as
\begin{equation}
	[G_{s}] = [\hat{\bm g}_{s_{1}} \cdots \hat{\bm g}_{s_{N}}]
\end{equation}
The null motion RW projection matrix $[\tau]$ is given by\cite{schaub}
\begin{equation}
	[\tau] = [I_{N\times N}] - [G_{s}]^{T} \left( [G_{s}] [G_{s}]^{T} \right)^{-1} [G_{s}]
\end{equation}
This project matrix maps any vector $\bm d$ into the null-space of $[G_{s}]$ such that no torque is exerted onto the spacecraft.  As a result, these null-motion solution never impact the stability or performance of the RW attitude control solution.  This concept is illustrated through:
\begin{align*}
	[G_{s}] [\tau] &= [G_{s}] \left( [I_{N\times N}] - [G_{s}]^{T} \left( [G_{s}] [G_{s}]^{T} \right)^{-1} [G_{s}] \right)
	\\
	&= [G_{s}] - [G_{s}] [G_{s}]^{T} \left( [G_{s}] [G_{s}]^{T}] \right)^{-1} [G_{s}] \\
	&= [G_{s}] - [G_{s}] \\
	&= [0_{N\times N}]
\end{align*}


\subsection{RW Wheel Speed Reduction Control}
Let $J_{s_{i}}>0$ be the RW inertia about the spin axis $\hat{\bm g}_{s_{i}}$, $\Omega_{i}$ be the RW spin speed, and $\dot{\bm \omega}$ be the inertial spacecraft angular acceleration vector.  Let $\Omega_{i,d}$ be the desired wheel speeds, and
\begin{equation}
	\Delta\Omega_i = \Omega_i - \Omega_{i,d}
\end{equation}
 be the wheel speed tracking errors.  
 
The RW motor torque equation is given by\cite{schaub}
\begin{equation}
	u_{s_{i}} = J_{s_{i}} (\dot\Omega_{i} + \hat{\bm g}_{s_{i}}^{T} \dot{\bm \omega} )
\end{equation}
Assuming that the spacecraft angular accelerations are much smaller than the wheel accelerations, this is approximated as
\begin{equation}
	u_{s_{i}} \approx J_{s_{i}} \dot\Omega_{i}
\end{equation}
Let $d_{i}$ be the desired torque to drive the $i^{\text{th}}$ RW spin rate $\Omega_{i}$  to the desired rate $\Omega_{i,d}$ be given through
\begin{equation}
	d_{i} = - K \Delta\Omega_{i}
\end{equation}
where the feedback gain $K>0$.  Then setting $u_{s_{i}} = d_{i}$ ideally provides the stable closed loop response
\begin{equation}
	J_{s_{i}} \dot\Omega_{i} + K\Delta\Omega_{i} = 0
\end{equation}
as $\dot\Omega_{i,d} = 0$.  



Let $\bm d$ be the $N\times 1$ array of desired RW decelerating motor torques given by
\begin{equation}
	\bm d = -K \Delta\bm\Omega
\end{equation}
If this RW motor torque were directly applied then a non-zero torque would be produced onto the spacecraft causing attitude deviations.  Instead, this desired despin torque $\bm d$ is mapped through $[\tau]$ onto the null space of the RW array using
\begin{equation}
	\bm u_{s,\text{null}} = [\tau] \bm d
\end{equation}

Assume the attitude feedback RW motor control solution is given by $\bm u_{s,\text{cont}}$, then final module RW motor torque array is the sum of these two torques.
\begin{equation}
	\bm u_{s} = \bm u_{s,\text{cont}} + \bm u_{s,\text{null}} 
\end{equation}




