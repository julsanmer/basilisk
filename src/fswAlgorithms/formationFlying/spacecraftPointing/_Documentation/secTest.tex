% !TEX root = ./Basilisk-spacecraftPointing-20190116.tex

\newpage

\section{Test Description and Success Criteria}

\subsection{Test 1}
This test that is performed in order to verify that the module works is a test that inputs a chief position vector and a deputy position vector for each timestep. The chief position input vector looks like the following.

\begin{lstlisting}
              [[np.cos(0.0), np.sin(0.0), 0.0],
              [np.cos(0.001), np.sin(0.001), 0.0],
              [np.cos(0.002), np.sin(0.002), 0.0],
              [np.cos(0.003), np.sin(0.003), 0.0],
              [np.cos(0.004), np.sin(0.004), 0.0]]
\end{lstlisting}

The chief position input vector looks like:

\begin{lstlisting}
               [[0.0, 0.0, 0.0],
               [0.0, 0.0, 0.0],
               [0.0, 0.0, 0.0],
               [0.0, 0.0, 0.0],
               [0.0, 0.0, 0.0]]
\end{lstlisting}

The chief thus makes a circle around the deputy. A total of three checks are performed. The first check checks whether ${\bm \sigma}_{R_{1}N}$ is in coherence with the expected value. The second check checks whether $^{\cal N}{\bm \omega}_{RN}$ is equal to the expected value and the final check checks for $^{\cal N}\dot{\bm \omega}_{RN}$. In case all three checks are successful, the module is considered working.

\subsection{Test 2}
In order to verify the robustness of the creation of a coordinate system around the vector defined by the user ($^{\cal B}{\bm a}$) the following input is used: [0,0,1]. This vector aligns with the z-axis of the deputy body frame. This would result in an undefined cross product between the z-axis of the body frame and $^{\cal B}{\bm a}$.  For this reason, in case these two vectors are aligned, the module takes the cross product between $^{\cal B}{\bm a}$ and the y-axis in the body frame. To test whether this is redundant, ${\bm \sigma}_{BA}$ is tested for this input.

\section{Test Parameters}

For an alignmentVector ($^{\cal B}{\bm a}$) that is equal to [1.0, 0.0, 0.0] the unit test verifies that the module output reference message vectors match expected values. Besides that, the second test verifies that ${\bm \sigma}_{BA}$ is indeed the value that is expected.

\begin{table}[ht]
\centering
\begin{tabular}{cc}
\hline
\hline
\textbf{Output Value Tested}     & \textbf{Tolerated Value} \\ \hline
${\bm \sigma}_{R_{1}N}$          & 1e-12                    \\
$^{\cal N}{\bm \omega}_{RN}$     & 1e-09                     \\
$^{\cal N}\dot{\bm \omega}_{RN}$ & 1e-12                     \\ 
${\bm \sigma}_{BA}$ & 1e-12  \\ \hline
\hline
\end{tabular}
\end{table}

\section{Test Results}

\begin{table}[H]
	\caption{Test results}
	\label{tab:results}
	\centering \fontsize{10}{10}\selectfont
	\begin{tabular}{c | c}
		\hline\hline
		\textbf{Check} 						  		&\textbf{Pass/Fail} \\ 
		\hline
	   Test 1	   			& \textcolor{ForestGreen}{PASSED} \\ 
	   Test 2	   			& \textcolor{ForestGreen}{PASSED} \\ 
	   \hline\hline
	\end{tabular}
\end{table}