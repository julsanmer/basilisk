% !TEX root = ./Basilisk-msisAtmosphere-20190221.tex

\section{Test Description and Success Criteria}
This section describes the specific unit tests conducted on this module.

\subsection{General Functionality}
The unit test check the neutral density calculation for a single spacecraft at a particular epoch time.  The epoch is either specified through an epoch input message, or set directly by specifying the {\tt epochDoy} day of year value.  

\subsection{Model-Specific Tests}

By design, this module shares much of its functionality with \verb|exponentialAtmosphere|; shared functionality is not checked by redundant tests.

\subsubsection{test\_unitTestNrlmsise00.py}
This integrated test evaluates the NRLMSISE-00 model at a given point in an orbit with zero'd (i.e., nonphysical) space weather inputs and verifies its outputs against the outputs of the base C model with identical inputs. This is a comprehensive test of the NRLMSISE-00 implementation in BSK, as it checks out the end-to-end functionality of the space weather messaging system, the geodetic conversion, and the interface to NRLMSISE-00 itself.

\section{Test Parameters}
The simulation tolerances are shown in Table~\ref{tab:errortol}.  In each simulation the neutral density output message is checked relative to python computed true values.  
\begin{table}[htbp]
	\caption{Error tolerance for each test.}
	\label{tab:errortol}
	\centering \fontsize{10}{10}\selectfont
	\begin{tabular}{ c | c } % Column formatting, 
		\hline\hline
		\textbf{Output Value Tested}  & \textbf{Tolerated Error}  \\ 
		\hline
		{Output Neutral Mass Density}        & {0.1 (relative) }  \\
		{Output Temperature}        & 0.1 (relative)  \\ 
		\hline\hline
	\end{tabular}
\end{table}




\section{Test Results}
The following two tables show the test results.  All tests are expected to pass.


\begin{table}[H]
	\caption{Test result for test\_unitMsisAtmosphere.py}
	\label{tab:results}
	\centering \fontsize{10}{10}\selectfont
	\begin{tabular}{c  | c  | c} % Column formatting, 
		\hline\hline
		{\tt orbitCase} & {\tt setEpoch} &  \textbf{Pass/Fail} \\ 
		\hline
		LPO & Direct &  \textcolor{ForestGreen}{PASSED} \\ 
		LTO & Direct &  \textcolor{ForestGreen}{PASSED} \\ 
		LPO & Msg &  \textcolor{ForestGreen}{PASSED} \\ 
		LTO & Msg &  \textcolor{ForestGreen}{PASSED} \\ 
		LPO & Default &  \textcolor{ForestGreen}{PASSED} \\ 
		LTO & Default &  \textcolor{ForestGreen}{PASSED} \\ 
		\hline
		\hline
	\end{tabular}
\end{table}


